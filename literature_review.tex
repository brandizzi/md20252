A solid understanding of Kubernetes is essential, so we rely primarily on \cite{luksa_kubernetes_nodate}, supplemented by \cite{kubernetes_authors_horizontal_2024}.

To account for Java Virtual Machine behavior, \cite{sullins_jmx_2002} serves as our main reference, particularly for internal (JMX) metrics and garbage collector behavior. Despite its age, it remains relevant for interpreting JVM metrics.

A cornerstone of our research is the survey by \cite{lorido-botran_review_2014}, which provides a comprehensive overview of auto-scaling techniques. Though older, it remains valuable alongside more recent surveys \cite{masdari_survey_2020, qu_auto-scaling_2018}.

While our broader goal includes exploring various approaches, we focus here on time series methods. Introductory works such as \cite{shumway_time_2017, siemsen_6_nodate} were particularly useful, while studies like \cite{yuan_using_2020, watada_emerging_2019} offered promising early results. Additionally, \cite{yuan_time_2024, dantas_autoscaling_2023} provide insights specific to Kubernetes, though their applicability to our objectives still requires assessment.

For this project, we adopt the Seasonal Autoregressive Integrated Moving Average with Exogenous Regressors (SARIMAX) model. The paper describing it \cite{mulla_times_2024} became a key reference, as SARIMAX is a powerful, widely supported time series model and effectively a superset of many alternatives.
